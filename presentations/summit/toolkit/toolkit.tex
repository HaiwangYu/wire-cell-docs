% TEXINPUTS=.:$HOME/git/bvtex: latexmk  -pdf <main>.tex
\documentclass[xcolor=dvipsnames]{beamer}

\input{defaults}
\input{beamer/preamble}

\setbeamertemplate{navigation symbols}{}
% \setbeamertemplate{background}[grid][step=1cm]

\usepackage{siunitx}
\usepackage{xmpmulti}
\usepackage[export]{adjustbox}
\usepackage{pifont}% http://ctan.org/pkg/pifont
\newcommand{\cmark}{\ding{51}}%
\newcommand{\xmark}{\ding{55}}%

\usepackage[outline]{contour}
\usepackage{tikz}
\usetikzlibrary{shapes.geometric, arrows}
\usetikzlibrary{positioning}

\definecolor{bvtitlecolor}{rgb}{0.98, 0.92, 0.84}
\definecolor{bvoutline}{rgb}{0.1, 0.1, 0.1}

\renewcommand{\bvtitleauthor}{Brett Viren}
\renewcommand{\bvtit}{Wire Cell Toolkit}
\renewcommand{\bvtitle}{\LARGE Wire Cell Toolkit\\Architecture and Development Status}
\renewcommand{\bvevent}{Wire Cell Summit 7-9 Dec 2015}
\renewcommand{\bvbeamerbackground}{}

% http://tex.stackexchange.com/a/23550
\makeatletter
\providecommand{\beamer@slideinframe}{0}%
\tikzset{highlight/.code={\ifnum#1=\beamer@slideinframe \tikzset{draw=red,text=red}\fi},highlight/.value required}
\makeatother

\setbeamertemplate{headline}{%
\leavevmode%
  \hbox{%
    \begin{beamercolorbox}[wd=\paperwidth,ht=2.5ex,dp=1.125ex]{palette quaternary}%
    \insertsectionnavigationhorizontal{\paperwidth}{}{\hskip0pt plus1filll}
    \end{beamercolorbox}%
  }
}
\begin{document}
\input{beamer/title.tex}
\input{beamer/toc.tex}

\lstset{%
  language=C++,
                basicstyle=\scriptsize\ttfamily,
                keywordstyle=\color{blue}\ttfamily,
                stringstyle=\color{red}\ttfamily,
                commentstyle=\color{gray}\ttfamily,
                morecomment=[l][\color{magenta}]{\#}
}

\section{Prototype vs. Toolkit}

\begin{frame}[fragile]
  \frametitle{WC Prototype vs. WC Toolkit}
  \begin{center}
    \begin{tabular}[h]{rcl}
      make it work & $\rightarrow$ & make it fast \\
      initial development & $\rightarrow$ & long-term improvements \\
      one developer & $\rightarrow$ & many developers \\
    \end{tabular}
  \end{center}

  Some commonalities and differences:
  \begin{itemize}
  \item C++11, explicit data models, various I/O
  \item portable, waf-based build, lives in GitHub.
  \end{itemize}

  \scriptsize
  \begin{center}
    \begin{tabular}[h]{|r|c|c|}
      \hline
      & Prototype & Toolkit \\
      \hline
      internal deps & tightly coupled & pervasive use of abstract interfaces \\
      ROOT & intimate &  independent (only tests + I/O libraries)\\
      \hline
      interface & many \texttt{main()}'s & API, single, general-purpose CLI \\
      execution & single threaded & abstract ``data flow programming'' engine \\
      configuration & hard coded & ``configurable'' interface, JSON files \\
      app construction & hard coded & DFP graph, dynamic plugin system \\
      \hline
      maintenance & Xin hacking! & long-term, multi-developer \\
      unit tests & some & many \\
      \hline
      \hline
      algorithms & state of the art & playing catch-up \\
      \hline
    \end{tabular}
  \end{center}

\end{frame}

\section{Design Goals}

\begin{frame}
  \tableofcontents[currentsection,hideothersubsections]
\end{frame}

\begin{frame}
  \frametitle{Toolkit Design Goals}

  Want the toolkit to:
  \begin{itemize}
  \item be portable to multiple ('nix) architectures: laptop/workstation, Grid, HPC, including GPU.
  \item support multiprocessing with (more or less) thread-unaware algorithms.
  \item dictate interface but not implementation.
  \item support multiple independent algorithm developers.
  \item encourage fine-grained unit testing.
  \item Provide cheap package creation and aggregation.
  \end{itemize}

\end{frame}


\section{Architecture}

\begin{frame}
  \tableofcontents[currentsection,hideothersubsections]
\end{frame}

\subsection{Concepts}

\begin{frame}
  \frametitle{Toolkit vs. Framework}
  \begin{center}
    \begin{tabular}[h]{|r|c|c|}
      \hline
      & Toolkit & Framework \\
      \hline
      \texttt{main()} & \xmark & \cmark \\
      \hline
    \end{tabular}
  \end{center}

  \vspace{3mm}

  Wire Cell Toolkit does:
  \begin{itemize}
  \item dictate transient data model and active class interfaces.
  \item provide a structure in which to implement functionality.
  \end{itemize}

  \vspace{3mm}

  Wire Cell Toolkit does not:
  \begin{itemize}
  \item determine user interface, 
  \item enforce an execution model,
  \item nor enforce file formats.
  \end{itemize}
  But it does provide some ``batteries included'' for all of these.
\end{frame}

\begin{frame}[fragile]
  \frametitle{High-level Wire Cell Toolkit Design Concepts}
  \begin{description}
  \item[interfaces] all toolkit components implement and communicate through abstract base classes.
  \item[data model] instances are accessed via const shared pointers to their interface class for safe memory management.
  \item[compute model] units (``nodes'') defined as interfaces consuming and producing data model interfaces.
  \item[factory] concrete interface instance construction via named lookup, supports dynamic plugins.
  \item[configurable] components may accept parameters from a unified configuration system.
  \item[application] component aggregation left to developer/user discretion or through toolkit facilities.
  \end{description}
\end{frame}


\begin{frame}[fragile]
  \frametitle{Interfaces}
  Interfaces define the \textit{verbs} and \textit{nouns} of the Wire Cell language.

  \vfill

  Data model examples:
  \begin{itemize}
  \item \href{https://github.com/WireCell/wire-cell-iface/blob/master/inc/WireCellIface/IData.h}{IData} transient data base class.
  \item \href{https://github.com/WireCell/wire-cell-iface/blob/master/inc/WireCellIface/IWire.h}{IWire},
    \href{https://github.com/WireCell/wire-cell-iface/blob/master/inc/WireCellIface/ICell.h}{ICell} defines wire/cell geometry
  \item \href{https://github.com/WireCell/wire-cell-iface/blob/master/inc/WireCellIface/IDepo.h}{IDepo}, 
    \href{https://github.com/WireCell/wire-cell-iface/blob/master/inc/WireCellIface/IDiffusion.h}{IDiffusion} simulation intermediates.
  \item \href{https://github.com/WireCell/wire-cell-iface/blob/master/inc/WireCellIface/IFrame.h}{IFrame}, 
    \href{https://github.com/WireCell/wire-cell-iface/blob/master/inc/WireCellIface/ITrace.h}{ITrace} defines waveform data
  \end{itemize}

  Compute unit examples:
  \begin{itemize}
  \item 
    \href{https://github.com/WireCell/wire-cell-iface/blob/master/inc/WireCellIface/ICellMaker.h}{ICellMaker},
    \href{https://github.com/WireCell/wire-cell-iface/blob/master/inc/WireCellIface/IDrifter.h}{IDrifter}, 
    \href{https://github.com/WireCell/wire-cell-iface/blob/master/inc/WireCellIface/IFramer.h}{IFramer}, 
    \href{https://github.com/WireCell/wire-cell-iface/blob/master/inc/WireCellIface/IChannelCellSelector.h}{ICellSelector} 
    are examples of transformative nodes 
    (eg, \href{https://github.com/WireCell/wire-cell-iface/blob/master/inc/WireCellIface/IBufferNode.h}{IBufferNode},
    \href{https://github.com/WireCell/wire-cell-iface/blob/master/inc/WireCellIface/IFunctionNode.h}{IFunctionNode}).
  \item 
    \href{https://github.com/WireCell/wire-cell-iface/blob/master/inc/WireCellIface/IWireSource.h}{IWireSource},
    \href{https://github.com/WireCell/wire-cell-iface/blob/master/inc/WireCellIface/IFrameSource.h}{IFrameSource} are 
    \href{https://github.com/WireCell/wire-cell-iface/blob/master/inc/WireCellIface/ISourceNode.h}{source nodes}.
    and \href{https://github.com/WireCell/wire-cell-iface/blob/master/inc/WireCellIface/ICellSliceSink.h}{ICellSliceSink}
    \href{https://github.com/WireCell/wire-cell-iface/blob/master/inc/WireCellIface/ISinkNode.h}{sink nodes}.
  \end{itemize}

  More interfaces exist and more to be added as we progress.

  \vfill
  
  All interface classes go in the package \href{https://github.com/WireCell/wire-cell-iface}{wire-cell-iface}.
\end{frame}

\begin{frame}[fragile]
  \frametitle{Named Factory Method and Plugins - using}

  NamedFactory provides string-to-object lookup supporting
  configuration and application levels.

  \begin{lstlisting}[language=C++]
string cname = "MyClass";
string iname = "my happy instance";
IMyInterface* obj = factory::lookup<IMyInterface>(cname, iname);
  \end{lstlisting}
  \begin{itemize}
  \item Find an instance implementing an interface by its concrete class \textbf{name} and an optional instance name.
  \item Behind the scenes, may use optional plugin system to check shared libraries for \texttt{MyClass}.
  \item Identical lookups return same instance, default-construct if not yet seen.
  \end{itemize}
\end{frame}

\begin{frame}[fragile]
  \frametitle{Named Factory Method and Plugins - Back end}
  \begin{lstlisting}[language=C++]
WIRECELL_NAMEDFACTORY_BEGIN(BoundCells)
WIRECELL_NAMEDFACTORY_INTERFACE(BoundCells, ICellMaker);
WIRECELL_NAMEDFACTORY_END(BoundCells)
  \end{lstlisting}

  \begin{itemize}
  \item Interface implementation must register with (singleton)
    factory at file scope to bind concrete class name to the interface
    it implements.
  \end{itemize}

  \vfill

  \begin{lstlisting}[language=C++]
PluginManager& pm = PluginManager::instance();
pm.add("WireCellGen");
  \end{lstlisting}

  \begin{itemize}
  \item Application or configuration layer must register shared
    libraries holding components with a (singleton) plugin manager.
  \end{itemize}
\end{frame}

\begin{frame}
  \frametitle{Configuration}

  This is still being developed so still partly conceptual:

  \begin{itemize}
  \item Configurable classes implement \texttt{IConfigurable} and register with \texttt{NamedFactory}.
  \item User supplies a JSON file containing a dictionary.
    \begin{itemize}
    \item[$\rightarrow$] keys map to concrete class/instance names.
    \item[$\rightarrow$] values follow target-specific schema.
    \end{itemize}
  \item Parsed and interpreted by a configuration manager.
  \item Data driven interpretation but some special cases:
    \begin{itemize}
    \item plugin manager must be configured early
    \item execution manager must be instantiated and configured
    \item execution manager configuration drives the rest.
    \end{itemize}
  \end{itemize}

  This dance must be done at application level but will be presented
  as a few high-level calls.
\end{frame}

\begin{frame}
  \frametitle{Application}

  It is up to the application developer to determine the scope and
  structure of how Wire Cell Toolkit components are called.

  \vfill

  Many Wire Cell Toolkit applications possible:

  \begin{itemize}
  \item A general-purpose \texttt{wire-cell} command line program is
    being developed and included with the toolkit.
  \item External framework modules may be created.
  \item A backend service is being considered in support of Bee 2.0.
  \end{itemize}

\end{frame}

\subsection{Data Flow Programming}

\begin{frame}
  \frametitle{Data Flow Programming}
  \footnotesize
  DFP structures the program as a graph.
  \vspace{-4mm}
  \begin{columns}
    \begin{column}{0.8\textwidth}
      \begin{description}
      \item[vertices] compute units (``nodes'')
      \item[edges] data queues
      \end{description}

      \begin{itemize}
      \item Thread safe queues + stateless nodes = ``easy'' parallel processing.
      \item Statefull still possible with concurrency=1 nodes.
      \item Nodes can be developed and tested in isolation.
      \item App-level programming by ``drawing'' the graph.
      \item Streamed processing can minimize RAM usage.
      \item Feedback loops may implement iterative flows.
      \item May instrument graph to collect performance data.
      \end{itemize}
      \flushright One possible example $\rightarrow$
    \end{column}

    \begin{column}{0.25\textwidth}
      \includegraphics[width=\textwidth]{dfp-exp.pdf}
    \end{column}
  \end{columns}
\end{frame}

\begin{frame}
  \frametitle{Abstract Execution Model}
  Wire Cell Toolkit defines abstract nodes and connection method:
  \begin{itemize}
  \item A node has zero or more input/output ``ports''.
  \item A port carries data of a specific (data model) type.
  \item A node declares its maximum concurrency.
  \item \href{https://github.com/WireCell/wire-cell-iface/blob/master/inc/WireCellIface/INode.h}{Node} and 
    \href{https://github.com/WireCell/wire-cell-iface/blob/master/inc/WireCellIface/IPort.h}{port} interface classes.
  \item ``Well known'' node types defined as mid-level interface classes. 
  \item
    \href{https://github.com/WireCell/wire-cell-iface/blob/master/inc/WireCellIface/IDataFlowGraph.h}{IDataFlowGraph}
    implements connection and graph execution methods.
  \item Battery included: \href{https://github.com/WireCell/wire-cell-tbb}{Intel TBB-based implementation}.
  \end{itemize}
  This works now but needs some final design polish.
\end{frame}

\section{Packages}

\begin{frame}
  \frametitle{Current package set}
  \footnotesize
  \texttt{wire-cell-}
  \begin{description}
  \item[\href{https://github.com/WireCell/wire-cell-build}{\texttt{build}}] 
    default source code aggregation and suite-wide build context.
  \item[\href{https://github.com/WireCell/wire-cell-util}{\texttt{util}}]
    NamedFactory, PluginManager, ConfigMangaer, Units, 3D~vector, special containers.
  \item[\href{https://github.com/WireCell/wire-cell-iface}{\texttt{iface}}] 
    data model and node interface classes.
  \item[\href{https://github.com/WireCell/wire-cell-gen}{\texttt{gen}}]
    simulation of 4/6 ``D''s: Deposit/Drift/Diffuse/Digitize \\
    (Detector response and Deconvolution still in development)
  \item[\href{https://github.com/WireCell/wire-cell-alg}{\texttt{alg}}]
    reference Wire Cell algorithms ported from prototype.
  \item[\href{https://github.com/WireCell/wire-cell-tbb}{\texttt{tbb}}]
    Intel TBB based DFP execution model implementation.
  \item[\href{https://github.com/WireCell/wire-cell-apps}{\texttt{apps}}]
    Provided end-user applications.
  \item[\href{https://github.com/WireCell/wire-cell-docs}{\texttt{docs}}]
    User/developer/installer manual.
  \item[\href{https://github.com/WireCell/wire-cell-sst}{\texttt{sst}}]
    Celltree file reading/writing.
  \item[\href{https://github.com/WireCell/wire-cell-bio}{\texttt{bio}}]
    Bee JSON file production.
  \item[\href{https://github.com/WireCell/wire-cell-rio}{\texttt{rio}}]
    Toolkit ROOT-based I/O.
  \item[\href{https://github.com/WireCell/wire-cell-rio}{\texttt{rootvis}}]
    ROOT-based visualization.
  \end{description}
  All are working at some level but are still in development.
\end{frame}

\begin{frame}
  \frametitle{A Wire Cell Toolkit Package}

  Three possible products from building a toolkit package:

  \footnotesize

  \begin{description}
  \item[lib] \texttt{inc/PackageName/*.h} and \texttt{src/*.cxx}
    turned into \\
    shared library+headers.
  \item[app] \texttt{apps/main-app.cxx} each source file made into an \\
    executable file ``\texttt{main-app}''.
  \item[test] \texttt{test/test\_*.cxx} each source file built to a
    test executable file and run as part of build each time code changes require it.
  \end{description}

  \begin{itemize}
  \item Each product type has own dependency tree (see next).

  \item There is a very low effort barrier to create new packages.

  \item Consider making a new package before adding to an existing one.

  \item Also can make new/personal build-aggregation packages to
    exercise narrower build contexts.

  \end{itemize}
\end{frame}

\begin{frame}[fragile]
  \frametitle{Package dependencies}
  \begin{columns}
    \begin{column}{0.4\paperwidth}
      \includegraphics[width=\textwidth]{pkg-deps-lib.pdf}
    \end{column}
    \begin{column}{0.2\paperwidth}
      \includegraphics[width=\textwidth]{pkg-deps-test.pdf}
    \end{column}
  \end{columns}

  \footnotesize
  \begin{itemize}
  \item No direct coupling among implementation libraries allowed, only loose via \texttt{iface}.
  \item \texttt{iface} provides ``simple'' data model implementation.
  \item Other packages provide data model imp to optimize memory/CPU (eg, lazy instantiating).
  \item Library and app dependencies strictly controlled, tests may violate.
  \end{itemize}
\end{frame}



\begin{frame}
  \frametitle{Hack on your own Wire Cell packages!}

  \begin{enumerate}
  \item Make a git repository in GitHub (or wherever you prefer).
  \item \texttt{mkdir -p inc/MyPackageName src test} and make a
    \texttt{wscript\_build} file based on
    \href{https://github.com/WireCell/wire-cell-alg/blob/master/wscript_build}{existing
      ones}.
  \item Fork
    \href{https://github.com/WireCell/wire-cell-build}{wire-cell-build}
    to add your package, or make a personal/reduced equivalent.
  \item Implement some existing DFP node interface class or work with
    me to develop new ones.
  \item If your node makes data, use existing ``simple data''
    (\href{https://github.com/WireCell/wire-cell-iface/blob/master/inc/WireCellIface/SimpleFrame.h}{eg})
    data model classes, subclass data model interfaces to implement
    your own or work with me to extend current data model.
  \item Write unit tests as you develop.
  \item Run a full-chain application with the \texttt{wire-cell}
    command line program (still in development).
  \end{enumerate}
\end{frame}

\section{Status and Summary}

\begin{frame}
  \tableofcontents[currentsection,hideothersubsections]
\end{frame}

\begin{frame}
  \frametitle{Current Full Chain}
  \begin{columns}
    \begin{column}{0.55\textwidth}
      \footnotesize
      \begin{itemize}
      \item Kinematics are just trivial straight-line ``tracks''.
      \item So far, focused mainly on simulation nodes.
        \begin{itemize}\scriptsize
        \item Still exercises all needed DFP features (sources, sinks, buffering and parallel)
        \item But with faster/simpler algorithms.
        \item Also for now, short-circuit \textit{detector response} + \textit{deconvolution} steps.
        \end{itemize}
      \item The only Wire Cell imaging algorithm so far is selecting potentially hit cells.
      \item \href{http://www.phy.bnl.gov/wire-cell/bee/set/18872eff-92ec-4e18-a8d0-645b9fc1e5aa/event/0/}{Sink node dumps a Bee file}.
      \item \href{https://github.com/WireCell/wire-cell-tbb/blob/master/test/test\_tbb\_dfp.cxx}{wire-cell-tbb/test/test\_tbb\_dfp.cxx}

      \end{itemize}
    \end{column}
    \begin{column}{0.5\textwidth}
      \vspace{-10mm}
      \includegraphics[width=\textwidth]{current-full-chain.pdf}
    \end{column}
  \end{columns}
\end{frame}

\begin{frame}
  \frametitle{Status}
  \footnotesize
  \begin{description}
  \item[\cmark\cmark\cmark] repos, build, packaging, dependencies all ``done'' \\
(will evolve as we port the prototype).
  \item[\cmark\cmark\xmark] initial data model established,\\
    needs to grow as more algorithms are ported.
  \item[\cmark\xmark\xmark] only the tiniest, first Wire Cell algorithm ported.  \\
    \textbf{Lots of work needed here, expect to do tuning/refactoring.}

  \item[\cmark\cmark\xmark] simulation needs 2 more ``D''s: detector response and deconvolution.
  \item[\cmark\cmark\xmark] parallel DFP working, but needs some small design tweaks.
  \item[\cmark\xmark\xmark] initial end-user configuration, straightforward to flesh out.
  \item[\cmark\xmark\xmark] celltree, Bee, native I/O needs fleshing out.

  \item[\cmark\xmark\xmark] general command line app started.\\
    Waiting on other progress (mostly dfp + cfg).

  \end{description}
\end{frame}

\begin{frame}
  \frametitle{Contributing to the Toolkit}
  Tookit development:
  \begin{itemize}\footnotesize
  \item Requires significant learning of current structure, good grasp
    of OO patterns, understanding OO vs GP paradigms, threading
    issues.  
  \item Possible overlap with active R\&D in the Gaudi world.
  \item I'd love some help/input here!
  \end{itemize}
  Implementing various needed DFP nodes:
  \begin{itemize}\footnotesize
  \item A basic understanding of the toolkit structure required.
  \item Porting algorithms from the prototype requires reading and
    understanding Xin's code (nontrivial but not impossible), looking
    for ways to factor it into well defined DFP nodes joined by well
    defined data.
  \item I/O modules need fleshing.  They just provide DFP nodes,
    mostly just a matter of typing in code.
  \end{itemize}
  Hacking your own ideas:
  \begin{itemize}\footnotesize
  \item Build the toolkit, start a package or two.
  \item Test out own interfaces for data and nodes.
  \item Work with me to incorporate changes.
  \end{itemize}
\end{frame}

\begin{frame}
  \frametitle{Summary}
  \begin{itemize}
  \item Wire Cell is transitioning from the \textbf{very successful working
    prototype} to a carefully designed toolkit to support long-term
    contributions from many developers, parallel processing and flexible
    integration with other systems.
  \item The toolkit supports the data-flow programming paradigm.
  \item An initial ``full chain'' (test) application exercises major toolkit functionality.
  \item Some structural work on the tookit itself is still needed so
    API is not yet fully stable, many fine-grained tests are developed.
  \item Much effort is needed to ``port'' prototype algorithms.
    \begin{itemize}
    \item[$\rightarrow$] contributions from others welcome and needed!
    \end{itemize}
  \end{itemize}
\end{frame}

\end{document}


%%% Local Variables:
%%% mode: latex
%%% TeX-master: t
%%% End:
