% TEXINPUTS=.:$HOME/git/bvtex: latexmk  -pdf <main>.tex
\documentclass[xcolor=dvipsnames]{beamer}

\input{defaults}
\input{beamer/preamble}

\setbeamertemplate{navigation symbols}{}
% \setbeamertemplate{background}[grid][step=1cm]

\usepackage{siunitx}
\usepackage{xmpmulti}
\usepackage[export]{adjustbox}
\usepackage{pifont}% http://ctan.org/pkg/pifont
\newcommand{\cmark}{\ding{51}}%
\newcommand{\xmark}{\ding{55}}%

\usepackage[outline]{contour}
\usepackage{tikz}
\usetikzlibrary{shapes.geometric, arrows}
\usetikzlibrary{positioning}

\definecolor{bvtitlecolor}{rgb}{0.98, 0.92, 0.84}
\definecolor{bvoutline}{rgb}{0.1, 0.1, 0.1}

\renewcommand{\bvtitleauthor}{Brett Viren}
\renewcommand{\bvtit}{Wire Cell / LArSoft Integration}
\renewcommand{\bvtitle}{\LARGE Wire Cell / LArSoft Integration}
\renewcommand{\bvevent}{Wire Cell Summit 7-9 Dec 2015}
\renewcommand{\bvbeamerbackground}{}

% http://tex.stackexchange.com/a/23550
\makeatletter
\providecommand{\beamer@slideinframe}{0}%
\tikzset{highlight/.code={\ifnum#1=\beamer@slideinframe \tikzset{draw=red,text=red}\fi},highlight/.value required}
\makeatother

\setbeamertemplate{headline}{%
\leavevmode%
  \hbox{%
    \begin{beamercolorbox}[wd=\paperwidth,ht=2.5ex,dp=1.125ex]{palette quaternary}%
    \insertsectionnavigationhorizontal{\paperwidth}{}{\hskip0pt plus1filll}
    \end{beamercolorbox}%
  }
}
\begin{document}
\input{beamer/title.tex}
\input{beamer/toc.tex}

\lstset{%
  language=C++,
                basicstyle=\scriptsize\ttfamily,
                keywordstyle=\color{blue}\ttfamily,
                stringstyle=\color{red}\ttfamily,
                commentstyle=\color{gray}\ttfamily,
                morecomment=[l][\color{magenta}]{\#}
}

\section{Overlap and Synergy}

\begin{frame}
  \frametitle{LArSoft and Wire Cell Characteristics Compared}

  \begin{center}
    \begin{tabular}[h]{|r|c|c|}
      \hline
      & LArSoft/Art & Wire Cell \\
      \hline
      arch & framework & toolkit \\
      exec & serial & parallel DFP (or DIY) \\
      sim & Geant4 + the 5 ``D''s &  just the 5 ``D''s\footnote{Deposit, Drift, Diffuse, Detector response, Digitize.} \\
      reco & mostly 2D then merge & 3D image + pat.rec. \\
      algs & alt/competing algs & one alg/step (no alt) \\
      chain & full MC + data & drift sim + data \\
      perf & fast and good & very slow but better \\
      build & fnal.gov  & portable \\
      support & SCD & shoestring \\
      maturity & adult & teenager \\
      \hline
    \end{tabular}
  \end{center}

\end{frame}

 %  graph of data exchange

\begin{frame}
  \frametitle{Opportunities for Data Exchange}
  \begin{columns}
    \begin{column}{0.5\textwidth}
      \footnotesize
      \begin{itemize}
      \item In principle, can exchange data between many steps in a
        fine-grained manner.
      \item Not all exchanges are profitable, eg, WC trajectories are
        trivial, LS 3$\times$2D imaging may not work with WC clustering.
      \item Solid \textcolor{red}{red}/\textcolor{blue}{blue} lines show likely fruitful exchange.
      \item Every line implies real effort:
        \begin{itemize}\scriptsize
        \item Data model matching.
        \item Converters.
        \item Exchange channel.
        \end{itemize}
      \end{itemize}
    \end{column}
    \begin{column}{0.5\textwidth}
      \includegraphics[width=\textwidth]{lswc-data-exchange.pdf}
    \end{column}
  \end{columns}
\end{frame}

\section{The Integration Challenge}

\begin{frame}
  \tableofcontents[currentsection,hideothersubsections]
\end{frame}

\begin{frame}
  \frametitle{Integration Challenges}
  The good:
  \begin{itemize}
  \item A file exchange method already in use (celltree).
  \item Wire Cell (toolkit) is explicitly designed for easy integration a the code level (compiling/linking)
  \end{itemize}
  Main challenges:
  \begin{description}
  \item[Memory] each of LS and WC alone require $\sim$2GB
    (individual memory usage is largely \textbf{not} redundant)
    \begin{itemize}\footnotesize
    \item Fermigrid limits to 2GB/job by default, other clusters similar.
    \item Massively parallel systems even more limited in RAM/core (thus shared-memory, multi-threaded WC developed).
    \end{itemize}
  \item[Portability] ROOT broken on HPC (at least on ANL Mira),
    general LArSoft/Art portability issues are known.
  \item[Expertise] need to understand different code bases, designs, data and execution models.
  \end{description}
\end{frame}

\begin{frame}
  \frametitle{Some Integration Considerations} 

  Challenges aside, we need \textit{some} form of LS/WC integration.  

  \vspace{3mm}

  My most important considerations:

  \begin{itemize}
  \item We must not impede ongoing development of LS nor WC.
    \begin{itemize}
    \item Each are separate projects with own development cultures.
    \end{itemize}
  \item Confront the realities of hardware architecture limitations
    particularly those needed for production processing.
    \begin{itemize}
    \item These are not just LS/WC or even WC-only challenges.
    \end{itemize}
  \item Facilitate comparison studies between all viable LArTPC reco
    methods.
    \begin{itemize}
    \item Integration for studies is likely very different than integration for production processing.
    \end{itemize}
  \end{itemize}


\end{frame}

\section{Some Integration Methods}

\begin{frame}
  \tableofcontents[currentsection,hideothersubsections]
\end{frame}

\begin{frame}
  \frametitle{Exchange Files}

  \begin{itemize}
  \item Run LArSoft and Wire Cell as separate executables.
  \item Pass data between the two via files in batch/serial manner.
  \item Incrementally adds to file bookkeeping overhead.
  \item Care in avoiding I/O bottlenecks, esp. in prod. proc.
  \item Effort to code converters, define file formats.
  \item Already have a good start with ``\textbf{celltree}'' files.
    \begin{itemize}\footnotesize
    \item Can hold: MC particle truth, energy dep. digitized waveforms, WC image (space/charge points) and....
    \item Supporting modules in both LS and WC (prototype and toolkit) 
    \end{itemize}
  \end{itemize}

  Notes:
  \begin{itemize}
  \item \textbf{celltree} file do not depend on neither WC nor LS
    code.  They hold POD + TClonesArray of TH1F.
  \item Can someone investigate use of \textbf{LArLite} files?
  \end{itemize}


\end{frame}

\begin{frame}
  \frametitle{Wire Cell in Single LArSoft Module}
  \begin{itemize}
  \item Write a WC ``application'' which is a LS module/algorithm.
  \item LS compiled/linked against WC libs, can still make use of WC configuration, plugin, execution managers.
  \item In principle, threading should be okay:
    \begin{itemize}\footnotesize
    \item LS thinks WC is just a function call.
    \item LS single threaded, gives control to WC. 
    \item WC threads do not run LS code (nor ROOT).
    \end{itemize}
  \item Single image may hit memory pressures on Grid/HPC.
  \item On HPC, hits ROOT  support problem and general LArSoft/Art portability problems.
  \item Maybe okay for doing comparison studies on high-memory,
    high-core workstations running Sci. Linux.
  \item Development of such a module requires all of FNAL software
    infrastructure support. (Not it!)
  \end{itemize}
\end{frame}

\begin{frame}
  \frametitle{Variation: Wire Cell in Many LArSoft Modules}
  A dubious idea:
  \begin{itemize}\footnotesize
  \item A WC DFP can be any size, even one node, or even call to a bare node.
  \item Can, in principle, distribute WC nodes across many LS modules.
  \item Won't help with memory pressures and reduces parallelism.
  \item Puts all intermediates into LS data store, but requires
    fine-grained data model development and converter coding.
  \end{itemize}
  It is a lot of work with no obvious benefit right now, but it is an option.
\end{frame}

\begin{frame}
  \frametitle{LArSoft client and Wire Cell Distributed Service}

  \begin{itemize}
  \item Write a LS module to make network request to a WC server.
  \item WC server is distributed on local cores/Cluster/Grid/HPC.
  \item The request transfers LS data to process.
  \item The response returns WC results.
  \item Synergy: need some a WC service to support Bee 2.0.
  \item For HPC: LS lives on high-memory/high-CPU but otherwise
    conventional ``edge'' computer.  This pattern is followed by other
    HPC uses (eg, ATLAS Event Service).
  \end{itemize}
\end{frame}


\begin{frame}
  \frametitle{Method Recommendations}
  My personal opinion of the ``right'' optima:
  \begin{center}
    \begin{tabular}[h]{|r|c|c|c|}
      \hline
      & files & module & client/server \\
      \hline
      reco dev & \cmark & & \\
      HPC dev & \cmark & & \cmark \\
      HPC prod &  & & \cmark \\
      Grid prod & \cmark & & \cmark \\
      compare & \cmark & & \\
      detailed & \cmark & \cmark & \\
      \hline
      Bee 2.0 &  & & \cmark \\
      \hline
    \end{tabular}
  \end{center}

  Notes:\scriptsize
  \begin{itemize}
  \item \textbf{Maybe} only detailed comparisons need to run WC as a LS module.
  \item Production processing memory pressure rules out WC as a LS module.
  \item I think HPC production ultimately requires client/server architecture.
  \item The backend service of Bee 2.0 likely to share many common
    components with what is used to support distributed production
    processing.
  \end{itemize}
\end{frame}

\section{FNAL Software Infrastracture}

\begin{frame}
  \tableofcontents[currentsection,hideothersubsections]
\end{frame}

\begin{frame}
  \frametitle{Integration into \texttt{fnal.gov}}
  Wire Cell prototype has been built at \texttt{fnal.gov}.
  \begin{itemize}\footnotesize
  \item half a day to locate and understand how to set up FNAL's ROOT6
    installation (mostly due to my ignorance and just passive time
    waiting for the right answer from experts on a mailing list).
  \item 10 minutes to download Wire Cell source, build and run its tests.
  \item Went according to instructions.  No expert tweaks needed despite Sci.~Linux not being the dev platform.
  \end{itemize}

  Microboone (Jyoti Joshi) is running Wire Cell prototype built in this manner on Fermi Grid nodes.
\end{frame}

\begin{frame}
  \frametitle{UPS'ify}
  Do we need UPS ``products'' made from Wire Cell prototype and/or toolkit?

  \vspace{3mm}

  Right now, I think ``no''.  Why?

  \begin{itemize}
  \item No formal Wire Cell releases yet (prototype nor toolkit).
  \item Would not be useful to WC developers
  \item Pure end-users are still almost nonexistent and they can build
    from source no prob.
  \item Long established, \texttt{fnal.gov}-based DUNE projects seem
    okay without UPS'ed experiment code (beam MC, Fast MC,
    mgt/globes).
  \end{itemize}

  So, it seems there is no rush.

  \vspace{3mm}

  If anyone wants to do it, go for it, but I think it is still too early to be useful.

\end{frame}


\section{Summary}

\begin{frame}
  \tableofcontents[currentsection,hideothersubsections]
\end{frame}

\begin{frame}
  \frametitle{Summary}
  \begin{itemize}
  \item Many options: lots of potential points and methods for integrating LArSoft and Wire Cell.
  \item Constraints: Integration strategy needs effort, needs
    to impede neither project, and is best directed toward most likely uses.
  \item Files: integration based on exchange files most needed now (and largely in place).
  \item Service: WC-as-a-service seems best for production running
    \begin{itemize}
    \item Needed for Bee 2.0 anyways, likely lots of overlap.
    \item Need a LS client module.
    \item \textbf{Any volunteers to work on either?}
    \end{itemize}
  \item UPS packaging of Wire Cell not expected to be needed right now.
  \end{itemize}
\end{frame}


\end{document}

%%% Local Variables:
%%% mode: latex
%%% TeX-master: t
%%% End:
